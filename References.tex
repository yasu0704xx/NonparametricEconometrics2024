\documentclass[a4paper,10pt]{jarticle}

\usepackage{graphicx, amssymb, amsmath, color, url, ulem}

\pagestyle{plain}

\title{Nonparametric Econometrics 2024}
\author{Yasuyuki Matsumura\thanks{M1 Student at Graduate School of Economics, Kyoto University. yasu0704xx \text{@} gmail.com.}}
\date{\today} %今日の日付が自動的に入る

%%%%%%    TEXT START    %%%%%%
\begin{document}
\maketitle % 文書のタイトルを生成

\section{教科書}
\begin{itemize}
  \item Li, Q. and J. S. Racine. (2007)
        \textit{Nonparametric Econometrics: Theory and Practice,} 
        Princeton University Press.

        \begin{itemize}
        \item 2024年度の計量経済学1, 2(by 西山慶彦先生)で輪読しています.シラバスはこちらからどうぞ:
              \url{https://www.k.kyoto-u.ac.jp/student/g/ec/support/syllabus_detail?no=5934}
        \item \sout{タイポが多くて,読み進めるのがやや面倒くさい.}
        \item \sout{Errataにない間違いも多数.}
        \item 過去には,下記のトピックコースでも使用されていたらしい.
        \item Bruce Hansen, ECON 718 NonParametric Econometrics (University of Wisconsin, Spring 2009).
              \url{https://users.ssc.wisc.edu/~bhansen/718/718.htm}
        \item 末石直也,セミ・ノンパラメトリック計量分析(京都大学,2014年度後期).
              \url{https://sites.google.com/site/naoyasueishij/teaching/nonpara?authuser=0}
        \item 各先生方のレクチャーノートは一般公開されている(2024年12月現在).
        \end{itemize}
  
\end{itemize}

\section{参考文献:英語のテキスト}

\begin{itemize}
  \item Hansen, B. E. (2022)
        \textit{Econometrics,} Princeton University Press.
        \begin{itemize}
          \item 19, 20, 21章でノンパラメトリック推定やregression Discontinuityの話を扱っている.
        \end{itemize}

  \item Horowitz, J. L. (2009)
        \textit{Semiparametric and Nonparametric Methods in Econometrics,}
        Springer.
        \begin{itemize}
          \item セミパラメトリック推定の話が一通りまとまっている.
        \end{itemize}
  
  \item van der Vaart, A. W. (2000) 
  \textit{Asymptotic Statistics,} Cambridge University Press.
        \begin{itemize}
          \item 数理統計学の超ド定番の教科書なので,詳細は省略.
          \item Chapters 24, 25がノンパラ,セミパラを扱っている.
        \end{itemize}

\end{itemize}


\section{参考文献:日本語のテキスト}

\begin{itemize}
  \item 久保木久孝,鈴木武 (2015) 『セミパラメトリック推測と経験過程』朝倉書店.
        \begin{itemize}
          \item セミパラというよりEmpirical Processの本っぽい.Glivenko-Cantelliとかでてくる.
        \end{itemize}  
  
  \item 清水泰隆 (2021) 『統計学への確率論,その先へ:ゼロからの測度論的理解と漸近理論への架け橋』内田老鶴圃.
        \begin{itemize}
          \item 測度論をひととおり勉強できる.優収束定理等の積分と極限の扱いを勉強するのに役立った.
        \end{itemize} 
    
  \item 清水泰隆 (2023) 『統計学への漸近論,その先は:現代の統計リテラシーから確率過程の統計学へ』内田老鶴圃.
        \begin{itemize}
          \item コアノメの副読本みたいな感じで読んでる.ノンパラは5章.
        \end{itemize}

  \item 末石直也 (2015) 『計量経済学:ミクロデータ分析へのいざない』日本評論社.
        \begin{itemize}
          \item ノンパラは9章で軽めに扱っている.
          \item パラメトリックの枠は出ないけど,分位点回帰,打ち切りモデル,Binary Choiceモデルなどなど,
                ノンパラ・セミパラで推定したいモデルの基礎がひととおり説明されている.
        \end{itemize}
  
  \item 末石直也 (2024) 『データ駆動型回帰分析:計量経済学と機械学習の融合』日本評論社.
        \begin{itemize}
          \item ノンパラ:3章,セミパラ:4章.
        \end{itemize}

  \item 西山慶彦,人見光太郎 (2023) 『ノン・セミパラメトリック統計解析(理論統計学教程:数理統計の枠組み)』共立出版.
        \begin{itemize}
          \item ややこしすぎる証明は元ペーパーを参照する形でカットされていて,読み進めやすい気がする.
        \end{itemize}
\end{itemize}

\section{参考文献:論文たち}

\begin{itemize}
  \item きりがないので省略.上記テキストの参考文献リスト等を見てください.
\end{itemize}




























\end{document}
