\documentclass[xcolor=svgnames,dvipdfmx,cjk]{beamer} 
\AtBeginDvi{\special{pdf:tounicode 90ms-RKSJ-UCS2}} 
\usetheme[progressbar=frametitle]{metropolis}
\setbeamertemplate{headline}[miniframes]
\setbeamercolor{background canvas}{bg=Snow} 
\setbeamercolor{frametitle}{fg=white, bg=gray!40!black}  
\setbeamercolor{section in head/foot}{fg=pink!70!red, bg=gray!30!black}
\setbeamercolor{alerted text}{fg=orange!50!red} 
\usefonttheme{professionalfonts}
\setbeamertemplate{theorems}[numbered]
\newtheorem{thm}{Theorem}[section]
\newtheorem{proposition}[thm]{Proposition}
\theoremstyle{example}
\newtheorem{exam}[thm]{Example}
\newtheorem{remark}[thm]{Remark}
\newtheorem{question}[thm]{Question}
\newtheorem{prob}[thm]{Problem}
\def\Avar{\text{Avar}}
\def\var{\text{var}}
\def\Var{\text{Var}}
\def\cov{\text{cov}}
\def\Cov{\text{Cov}}
\def\med{\text{med}}
\def\E{\mathbb{E}}
\def\R{\mathbb{R}}
\def\P{\mathbb{P}}
\def\parrow{\xrightarrow{p}}
\def\darrow{\xrightarrow{d}}
\def\plim{\text{plim }}
\usepackage{bbm}
\usepackage{ascmac}
\usepackage{tcolorbox}

%%%%%%ナビゲーションバー%%%%%%%%%%%%%%%%%%%%%%%%%%%%%%%%%%%%%%%%%%%%
%\useoutertheme{miniframes}
%\setbeamertemplate{headline}[miniframes theme]
%\setbeamertemplate{mini frames}[circle]

%\setbeamertemplate{headline}
%{%
%  \begin{beamercolorbox}{section in head/foot}
%    \vskip2pt\insertnavigation{\paperwidth}\vskip2pt
%  \end{beamercolorbox}%
%}
%%%%%%%%%%%%%%%%%%%%%%%%%%%%%%%%%%%%%%%%%%%%%%%%%%%%%%%%%%%%%%%%%%%

\begin{document} 

%%%%%講演に関する情報%%%%%%%%%%%%%%%%%%%%%%%%%%%%%%%%%%%%%%%%
\title{Censored Models} 
\subtitle{Li and Racine (2007, Chapter 11)}
\author{Yasuyuki Matsumura (yasu0704xx [at] gmail.com)}
\institute{Graduate School of Economics, Kyoto University}
\date{\today} % 日付を自動で挿入

%%%%%タイトルページ%%%%%%%%%%%%%%%%%%%%%%%%%%%%%%%%%%%%%%%%%%
\begin{frame}                  
\titlepage            
\end{frame}


%%%%%%%%%%%%%%%%%%%%%%%%%%%%%%%%%%%%%%%%%%%%%%%%%%%%%%%%%%%%%%%
%%%%%%%%%%%%%%%%%%%%%%%%%%%%%%%%%%%%%%%%%%%%%%%%%%%%%%%%%%%%%%%
%%%%%%%%%%%%%%%%%%%%%%%%%%%%%%%%%%%%%%%%%%%%%%%%%%%%%%%%%%%%%%%

\section{Parametric Censored Models}

\begin{frame}{Type-1 Tobit Model}
      \begin{itemize}
            \item Consider the following latent variable model:
                  \begin{align*}
                        Y_i^{\star} = X_i^T \beta + \epsilon_i, \quad i=1, \cdots n,
                  \end{align*}
                  where $X_i \in \R^q$ is an explanatory vector, 
                  $\beta$ is a $q \times 1$ vector of coefficients, and
                  $\epsilon_i$ is a mean zero disturbance term.
            \item $Y_i^{\star}$ is a latent variable, which we cannot observe.
                  Instead, we observe 
                  \begin{align*}
                        Y_i   &= Y_i^{\star} 1(Y_i^{\star} > 0) \\
                              &= \max\{ X_i^T \beta + \epsilon_i, \, 0 \}.
                  \end{align*}
            \item Note that the ``cutoff'' is set equal to $0$ without loss of generality.
                  That is, we expect that $Y_i$ ($\epsilon_i$) is censored at $0$ (resp. $-X_i^T\beta$). 
      \end{itemize}
\end{frame}

\begin{frame}{Parametric Approach}
      \begin{itemize}
            \item Popular parametric approaches include MLE and Heckit.
                  \footnote{Amemiya (1984):Tobitモデルのサーベイ論文;Amemiya (1985):教科書.}  
            \item These approaches demand the following distributional assumpution:
                  \begin{align*}
                        \epsilon_i | X_i \sim \text{Normal}(0, \sigma^2).
                  \end{align*}
                  Since $Y_i^{\star}$ is censored, for example, by top coding, 
                  the distribution of $Y_i^{\star}$ cannot be identified 
                  without this assumpution.
            \item In other words, these parametric approaches do not allow for the heteroscedasticity of $\epsilon_i$
                  (Arabmazar and Schmidt 1981).
      \end{itemize}  
\end{frame}

%%%%%%%%%%%%%%%%%%%%%%%%%%%%%%%%%%%%%%%%%%%%%%%%%%%%%%%%%%%%%%%
\section{Semiparametric Type-1 Tobit Models}

\begin{frame}{Semiparametric Type-1 Tobit Models}
      \begin{itemize}
            \item We introduce the following semiparametric type-1 Tobit model:
                  \begin{align*}
                        Y_i^{\star} &= X_i^T \beta + \epsilon_i, \\
                        Y_i         &=  Y_i^{\star} 1( Y_i^{\star} > 0).
                  \end{align*}
            \item For identifying the moments of $Y_i^{\star}$, 
                  we need additional assumputions.
            \item Powell (1984) proposes to assume that 
                  $\med(\epsilon_i|X_i) = 0$.
            \item Chen and Khan (2000) proposes a estimation procedure 
                  which requires weaker assumputions for identification than Powell (1984).
      \end{itemize}
\end{frame}


%%%%%%%%%%%%%%%%%%%%%%%%%%%%%%%%%%%%%%%%%%%%%%%%%%%%%%%%%%%%%%%

\section{Semiparametric Censored Regression Models}

\begin{frame}{Powell (1984): CLAD}
      \begin{itemize}
            \item Consider the semiparametric type-1 Tobit model:
                  \begin{align*}
                        Y_i^{\star} &= X_i^T \beta + \epsilon_i, \\
                        Y_i         &=  Y_i^{\star} 1( Y_i^{\star} > 0) = \max\{Y_i^{\star}, 0\}.
                  \end{align*}
            \item Assume that 
                  $ \med(\epsilon_i|X_i) = 0 $.
                  Noting that the ``monotonicity'' of median
                  \footnote{maxとmedの順番を入れ替えても大丈夫ということ.maxでなくても,単調変換なら入れ替え可.},
                  we obtain 
                  $\med(Y_i|X_i) = \max\{ \med(Y_i^{\star}|X_i), 0\} = \max\{X_i^T\beta, 0\}$,
                  which implies that the above model can be rewritten as
                  \begin{align*}
                        Y_i &= \max\{X_i^T \beta, 0\} + \epsilon_i, \\
                        \med(\epsilon_i|X_i) &= 0.
                  \end{align*}
      \end{itemize}
\end{frame}

\begin{frame}
      \begin{itemize}
            \item Powell (1984) proposes the following 
                  \alert{censored least absolute deviations estimator}:
                  \begin{align*}
                        \hat{\beta}_{clad} 
                              &= \arg\min_{\beta} \frac{1}{n} \sum_{i=1}^n |Y_i - \max\{X_i^T\beta , 0\}| \\
                              &= \arg\min_{\beta} \frac{1}{n} \sum_{i=1}^n 1(X_i^T\beta > 0)|Y_i - X_i^T \beta|.
                  \end{align*}
            \item Computation is sometimes complex
                  \footnote{$\beta$が2つの役割をもつことに起因する:データの選択,係数の値の決定.}.
                  See Buchinsky (1994); Khan and Powell (2001).
            \item Powell (1984) establishes the $\sqrt{n}$-consistency and asy. normality:
                  \begin{align*}
                        \sqrt{n} (\hat{\beta}_{clad} - \beta) \darrow \text{Normal} (0, V_{clad}^{-1}),
                  \end{align*}
                  where 
                  \begin{align*}
                        V_{clad} = 4 f^{2}(0) \E[1(X_i^T\beta >0) X_i X_i^T ]
                  \end{align*}
                  and $f(0)$ is the density of $\epsilon_i$ at the origin.
      \end{itemize}
\end{frame}

\begin{frame}
      \begin{itemize}
            \item Variance estimation can be implemented as follows.
            \item Assume that $\epsilon_i$ is independent of $X_i$.
            \item Note that 
                  \begin{align*}
                        f(0) &= \lim_{h \to 0} \P(0 \leq \epsilon_i < h) \\
                             &= \lim_{h \to 0} \P(0 \leq \epsilon_i < h | X_i^T\beta>0).
                  \end{align*}
            \item Powell suggests to estimate $f(0)$ by
                  \begin{align*}
                        \hat{f}(0) = \frac{
                              1(X_i^T \hat{\beta}_{clad} > 0) 1(0 \leq \hat{\epsilon}_i < h)
                        }{
                              h \sum_{i=1}^n 1(X_i^T \hat{\beta}_{clad} > 0)
                        }.
                  \end{align*}
      \end{itemize}
\end{frame}

\begin{frame}{\alert{Extension 1}: Estimation of $f(0)$}
      \begin{itemize}
            \item Horowitz and Neumann (1987) propose an alternative estimator of $f(0)$.
            \item To estimate $f(0)$, they use data with 
                  $X_i^T \hat{\beta}_{clad} \in \left[ -\frac{h}{2}, \frac{h}{2} \right]$.
            \item Their estimator is given by 
                  \tiny{
                  \begin{align*}
                        \hat{f}(0) = \frac{
                              \sum_{i=1}^n 
                              1\left(-\frac{h}{2}\leq \hat{\epsilon}_i \leq \frac{h}{2}\right) 
                              1\left(Y_i > 0\right)
                        }{
                              h \left[
                                    \sum_{i=1}^n 
                                    1(X_i^T \hat{\beta}_{clad} > \frac{h}{2}) 
                                    + \frac{1}{2} \left( 1 + \frac{X_i^T \hat{\beta}_{clad}}{\frac{h}{2}}\right)
                                    1\left(-\frac{h}{2} < X_i^T \hat{\beta}_{clad} \leq \frac{h}{2}\right)
                              \right].
                        }
                  \end{align*}
                  }
            \item Hall and horowitz (1990) suggest to replace the indicator function 
                  by a kernel function.
      \end{itemize}
\end{frame}

\begin{frame}{\alert{Extension 2}: Newey and Powell (1990)}
      \begin{itemize}
            \item Newey and Powell (1990) modify the objective function above:
                  \begin{align*}
                        \hat{\beta}_{np} 
                              = \arg\min_{\beta} 
                                \sum_{i=1}^n
                                w_i |Y_i - \max\{X_i^T \beta, 0\}|.
                  \end{align*}
            \item They show that the optimal weight is 
                  $w_i = 2 f(0|X_i)$.
                  The asy. variance is $\{ 4\E[1(X_i^T\beta>0) f^2(0|X_i)X_iX_i^T] \}^{-1}$.
            \item Their estimator achieves \alert{the semiparametric efficiency bound} 
                  for the censored regression model under $\med(\epsilon_i|X_i)=0$.
            \item If $\epsilon_i$ is independent of $X_i$, then $f(0|X_i) = f(0)$, 
                  which implies that $\hat{beta}_{np} = \hat{\beta}_{clad}$.
      \end{itemize}      
\end{frame}

\begin{frame}{\alert{Extension 3}: Other Approaches}
      \begin{itemize}
            \item Powell (1986): Additionally assume the symmetry assumption.
            \item Newey (1991): GMM-based estimation. Assume the symmetry assumption for efficiency.
            \item Honore and Powell (1994): Identically CLAD; Identically censored least squares.
      \end{itemize}
\end{frame}

%%%%%%%%%%%%%%%%%%%%%%%%%%%%%%%%%%%%%%%%%%%%%%%%%%%%%%%%%%%%%%%
\section{Nonparametric Heteroscedasticity}

\begin{frame}{Problems Arising with Powell's CLAD}
      \begin{itemize}
            \item Recall that Powell's CLAD requires 
                  $\med(\epsilon_i|X_i)=0$,
                  which can be interpreted as restrictive
                  \footnote{とはいえ,中央値の識別は,期待値の識別よりもはるかに緩い条件で済むので,
                  CLADやそれを拡張した打ち切りデータに対する分位点回帰をやろうという話になる.}
                  .
            \item $\Avar(\hat{\beta}_{clad})$ is represented using 
                  $\E[1(X_i^T\beta>0)X_i X_i^T]^{-1}$,
                  which cannot be defined if $\E[1(X_i^T\beta>0)X_i X_i^T]$ is not of full rank.
                  This problem often arises under heavy censoring
                   (i.e., when $X_i^T\beta$ is negative with high probability).
      \end{itemize}
\end{frame}


\begin{frame}{Chen and Khan (2000)}
      \begin{itemize}
            \item Chen and Khan (2000) consider estimation procedures for 
                  heteroscedastic censored linear regression models.
            \item Their approach requires weaker identification conditions than Powell's CLAD.
            \item They also allow for various degrees of censoring.
            \item Their main idea is that \alert{they model the error term as the product of 
                  a homoscedastic error and a scale function of $X_i$ that can be estimated using kernel methods}.
      \end{itemize}
\end{frame}

\begin{frame}
      \begin{itemize}
            \item They assume that 
                  \begin{align*}
                        &\epsilon_i = \sigma(X_i)v_i, \\
                        &\P(v_i \leq \lambda|X_i) \equiv \P(v_i \leq \lambda) \text{ for any } \lambda \in \R , X_i \, a.s.,\\
                        &\E(v_i)=0, \Var(v_i)=1. 
                  \end{align*}
            \item Recalling that $Y_i = \max\{X_i^T\beta + \epsilon_i, 0\}$, we obtain
                  \begin{align*}
                        &\text{For any } \alpha \in (0,1),\\
                        &q_{\alpha}(X_i) = \max\{X_i^T\beta + c_{\alpha}\sigma(X_i), 0\},
                  \end{align*}
                  where
                  \begin{align*}
                        & q_{\alpha}(\cdot) \text{ denotes the } \alpha \text{-th quantile of } Y_i \text{ given } X_i, \\
                        & c_{\alpha} \text{ denotes the } \alpha \text{-th quantile from the (unknown) distribution of } v_i.
                  \end{align*}
            \item Thus, for any $q_{\alpha_j}(X_i) > 0$ for two distinct $\alpha_1 \neq \alpha_2$, we have
                  \begin{align*}
                        q_{\alpha_j}(X_i) = X_i^T \beta + c_{\alpha_j} \sigma(X_i) \text{ for } j=1,\,2.
                  \end{align*}
      \end{itemize}
\end{frame}

\begin{frame}{Chen and Khan (2000): Estimation}
      \begin{itemize}
            \item Chen and Khan (2000) propose two estimators of $\beta$. 
                  One is assuming that $v_i$ has a known parametric distribution.
                  The other does not require such assumptions.
            \item Here we focus on the latter one.
            \item Notations:
                  \begin{align*}
                        \bar{q}_{\alpha}(\cdot) &= \frac{q_{\alpha_2}(\cdot) + q_{\alpha_1}(\cdot)}{2}, \\
                        \Delta q_{\alpha}(\cdot) &= q_{\alpha_2}(\cdot) - q_{\alpha_1}(\cdot), \\
                        \bar{c} &= \frac{c_{\alpha_2} + c_{\alpha_1}}{2}, \\
                        \Delta c &= c_{\alpha_2} - c_{\alpha_1}, \\
                        \gamma_1 &= \frac{\bar{c}}{\Delta c} \text{: we treat } \gamma_1 \text{ as a nuisance parameter.}
                  \end{align*}
            \item From $ q_{\alpha_j}(X_i) = X_i^T \beta + c_{\alpha_j} \sigma(X_i)$,
                  one can show that 
                  \begin{align*}
                        \bar{q}_{\alpha}(X_i) = X_i^T \beta + \gamma_1 \Delta q_{\alpha}(X_i) \text{ for } j=1,2.
                  \end{align*}
      \end{itemize}
\end{frame}

\begin{frame}
      \begin{itemize}
            \item Chen and Khan (2000)'s estimation procedures include the following steps:
            \item \alert{1st step}: Estimate $q_{\alpha_j}(\cdot)$ nonpaeametrically. 
                  Let $\hat{q}_{\alpha_j}(\cdot)$ denote the nonparametric estimator of $q_{\alpha_j}(\cdot)$.
            \item \alert{2nd step}: Regress $\hat{\bar{q}}_{\alpha}(\cdot)$ on $X_i$ and $\Delta \hat{q}_{\alpha}(\cdot)$.
            \item That is, the estimators of $\beta$ and $\gamma_1$ are given by minimizing (w.r.t. $\beta$ and $\gamma_1$)
                  \begin{align*}
                        \frac{1}{n} 
                        \sum_{i=1}^n 
                        \tau(X_i) 
                        w\left( \hat{q}_{\alpha_1}(X_i) \right)
                        [ \hat{\bar{q}}(X_i) - X_i^T\beta - \gamma_1 \Delta \hat{q}_{\alpha}(X_i) ]^2 ,
                  \end{align*}
                  where $w(\cdot)$ is a smoothing weight function
                  \footnote{$1( \hat{q}_{\alpha_1}(X_i)>0)$のかわりのようなもの.},
                  $\tau(\cdot)$ is a trimming function having compact support.
            \item Under certain regularity conditions, 
                  their estimator $\hat{\beta}$ have the parametric $\sqrt{n}$ rate of convergence,
                  and is distributed asymptotically normally.
      \end{itemize}
\end{frame}

\begin{frame}{\alert{Extension}: Cosslett (2004)}
      \begin{itemize}
            \item Cosslett (2004) proposes asymptotically efficient likelihood-based semiparametric estimators 
                  for censored and truncated regression models.
            \item See the paper for details. 
      \end{itemize}
\end{frame}

%%%%%%%%%%%%%%%%%%%%%%%%%%%%%%%%%%%%%%%%%%%%%%%%%%%%%%%%%%%%%%%

\section{The Univariate Kaplan-Meier CDF Estimator}

\begin{frame}{Kaplan and Meier (1958): Product-Limit Estimator}
      \begin{itemize}
            \item There exists a class of semiparametric estimators 
                  that employ the so-called Kaplan-Meier estimator of a CDF
                  in the presence of censored data.
            \item \alert{Setup}: Consider the following estimands:
                  \begin{align*}
                        \text{CDF: }& F(\cdot), \text{ or}\\
                        \text{Survival function: }& S(\cdot) = 1 - F(\cdot).
                  \end{align*}
            \item Let $\{Y_i\}_{i=1}^n$ be the random sample of interest drawn from $F(\cdot)$.
            \item Let $\{L_i\}_{i=1}^n$ be random/fixed censoring variables, that are independent of $\{Y_i\}_{i=1}^n$.
            \item Define $Z_i = \min\{Y_i, L_i\}$, and $\delta_i = 1(Y_i \leq L_i)$. 
                  Suppose that we observe only $Z_i$ and $\delta_i$. 
                  By construction, 
                  we cannot observe the exact value of $Y_i$ if $\delta_i = 0$.
      \end{itemize}
\end{frame}

\begin{frame}{cont'd}
      \begin{itemize}
            \item Define the ascending points $c_0, c_1, \cdots, c_m$ 
                  at which the CDF $F(\cdot)$ or $S(\cdot)$ is to be evaluated.
            \item Define $I_j = 1(Y > c_j)$.
            \item Noting that $c$'s are ascending 
                  and so that $I_{j-1} = 1$ if $I_j = 1$, 
                  we obtain conditional survival probability:
                  \begin{align*}
                        \P(I_j=1 | I_{j-1}=1) 
                              = \frac{\P(I_j=1)}{\P(I_{j-1}=1)}
                              = 1 - \frac{\P(c_{j-1} < Y \leq c_j)}{\P(Y>c_{j-1})}.
                  \end{align*}
            \item By choosing $c_0$ small enough (say, below the smallest observation in the data),
                  we can always ensure that $\P(I_0=1)=1$. 
                  That is, all items survive initially.
      \end{itemize}
\end{frame}

\begin{frame}{Estimation: In the Case of No Censoring}
      \begin{itemize}
            \item We can estimate $\P(I_j=1 | I_{j-1}=1)$ by the iteration of
                  \begin{align*}
                        \tilde{\P}(I_j=1|I_{j-1}=1)
                              &= \frac{\tilde{\P}(I_j=1)}{\tilde{\P}(I_{j-1}=1)} 
                               = \frac{\# \text{ of } Y_i > c_j}{\# \text{ of } Y_i > c_{j-1}} \\
                              &= 1- \frac{\# \text{ of } c_{j-1} < Y_i \leq c_j}{\# \text{ of } Y_i > c_{j-1}},
                  \end{align*}
                  which leads to the following estimator of survival probability:
                  \begin{align*}
                        \tilde{\P}(I_j=1) 
                              &= \prod_{s=1}^j \tilde{\P}(I_s=1 | I_{s-1}=1) \\
                              &= \frac{\# \text{ of } Y_i > c_j}{\# \text{ of } Y_i > c_0} = \frac{\# \text{ of } Y_i > c_j}{n} 
                               = 1 - \hat{F}^n(c_j)
                  \end{align*}
                  where $\hat{F}^n(c_j) = \dfrac{\# \text{ of } Y_i \leq c_j}{n}$ is the empirical CDF
                  \footnote{テキストでは$s=2$から計算することになっているが,このスライドでは,$c_0, \cdots$という点の取り方にconsistentな表記に統一した.}.
      \end{itemize}
\end{frame}

\begin{frame}{Estimation: With the Presence of Censoring}
      \begin{itemize}
            \item Similar estimation procedures to above can be implemented:
                  Iteration of 
                  \begin{align*}
                        \hat{\P}(I_j=1|I_{j-1}=1) = 1 -  \frac{\# \text{ of uncensored } c_{j-1} < Y_i \leq c_j}{\# \text{ of } Y_i > c_{j-1}}
                  \end{align*}
                  leads to the following estimator of survival probability:
                  \begin{align*}
                        \hat{S}(c_j) = \hat{\P}(I_j=1) = \prod_{s=1}^{j} \hat{\P}(I_s=1 | I_{s-1}=1).
                  \end{align*}
            \item The estimator of CDF is given by $\hat{F}(c_j) = 1 -  \hat{S}(c_j)$
                  \footnote{Errataをみると,$s=2$から計算することになっているが,このスライドでは,$c_0, \cdots$という点の取り方にconsistentな表記に統一した.}.
      \end{itemize}
\end{frame}


\begin{frame}{Disclaimer}
      \begin{center}
            以降の内容は手書きのノートで替えさせていただきます.

            宿題や試験に追われスライドが間に合いませんでした.
            
            余裕があったら春休みの間にBeamerにします...
      \end{center}
\end{frame}


%%%%%%%%%%%%%%%%%%%%%%%%%%%%%%%%%%%%%%%%%%%%%%%%%%%%%%%%%%%%%%%

\section{The Multivariate Kaplan-Meier CDF Estimator}






%%%%%%%%%%%%%%%%%%%%%%%%%%%%%%%%%%%%%%%%%%%%%%%%%%%%%%%%%%%%%%%

\section{Nonparametric Censored Regression}










%%%%%%%%%%%%%%%%%%%%%%%%%%%%%%%%%%%%%%%%%%%%%%%%%%%%%%%%%%%%%%%





\end{document}